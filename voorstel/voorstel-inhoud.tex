%---------- Inleiding ---------------------------------------------------------
\section{Inleiding}%
\label{sec:inleiding}

Deze bachelorproef richt zich op het ontwikkelen van een geïntegreerde applicatie die automatisch programmeercode en SQL-query's analyseert, opschoont en optimaliseert. In moderne softwareprojecten ontstaan door de omvang en complexiteit van code regelmatig inefficiënties, fouten en inconsistenties, wat leidt tot hogere ontwikkelkosten, trager presterende systemen en moeilijk onderhoud. Dit probleem is vooral relevant voor softwareontwikkelaars binnen middelgrote en grote IT-teams die regelmatig werken met Python en relationele databases.  

De centrale onderzoeksvraag van deze bachelorproef luidt: \textit{In welke mate kan een applicatie die automatische analyse en optimalisatie van code en SQL uitvoert, bijdragen aan het verbeteren van codekwaliteit en het reduceren van ontwikkeltijd in softwareprojecten?} Om deze vraag te beantwoorden worden zowel deelvragen met betrekking tot het probleemdomein als het oplossingsdomein geformuleerd. Wat betreft het probleemdomein wordt onderzocht welke veelvoorkomende problemen, fouten en inefficiënties voorkomen in Python-code binnen middelgrote en grote IT-teams, welke inefficiënties optreden in SQL-query's in relationele databases en wat de gevolgen zijn van slecht gestructureerde code en suboptimale SQL-query's voor onderhoud, ontwikkeltijd en systeemprestaties. Wat betreft het oplossingsdomein richt het onderzoek zich op het identificeren van bestaande tools en analysemethoden die fouten, code smells en inefficiënties in Python-code detecteren en corrigeren, welke hulpmiddelen beschikbaar zijn voor het opsporen en optimaliseren van inefficiënte SQL-query's, en hoe een geïntegreerde applicatie kan worden ontworpen die zowel Python-code als SQL-query's simultaan analyseert en concrete verbeteringen voorstelt.

Het doel van het onderzoek is het ontwikkelen van een proof-of-concept applicatie die ontwikkelaars ondersteunt bij het opsporen van veelvoorkomende fouten, inefficiënte patronen en suboptimale SQL-query's. Het concrete eindresultaat van deze bachelorproef omvat zowel een werkend prototype als een gedocumenteerde evaluatie van de effectiviteit van het systeem. Deze applicatie biedt inzicht in verbeterpunten en genereert gestructureerde aanbevelingen, waardoor de tijd die ontwikkelaars besteden aan handmatige code-review en foutopsporing wordt verminderd en de functionaliteit en betrouwbaarheid van de code behouden blijven. De inleiding vormt zo de basis voor de literatuurstudie, waarin relevante inzichten uit het vakgebied worden besproken, en voor de methodologie, waarin de concrete stappen en onderzoekstechnieken worden beschreven die nodig zijn om zowel de hoofd- als de deelvragen te beantwoorden.


%---------- Stand van zaken ---------------------------------------------------
\section{Literatuurstudie}%
\label{sec:literatuurstudie}

De literatuur toont aan dat zowel programmeercode als SQL-query's gevoelig zijn voor terugkerende problemen zoals syntaxfouten, code smells, inefficiënties en inconsistenties. Slecht gestructureerde code verhoogt de technische schuld, vertraagt het ontwikkelproces en leidt tot hogere onderhoudskosten \autocite{Guo2011,Zazworka2013}. Voor programmeercode blijken tools zoals linters (Pylint, Flake8), formatters (Black, Prettier) en statische analysetools (SonarQube, mypy) effectief in het detecteren van syntax- en typefouten, code smells en anti-patterns \autocite{Campbell2013,TanBockisch2019}. Voor SQL-query's bestaan soortgelijke hulpmiddelen, zoals sqlparse en sqlfluff, die inefficiënte structuren, ontbrekende filters en potentieel foutieve joins kunnen detecteren \autocite{Muse2020,OWASP2021}.  

Hoewel deze tools afzonderlijk nuttig zijn, ontbreekt er vaak een geïntegreerde workflow die zowel Python-code als SQL-query's simultaan analyseert en automatisch verbetert. Dit biedt ruimte voor onderzoek naar de ontwikkeling van een applicatie die deze verschillende analysetechnieken combineert en ontwikkelaars ondersteunt met concrete aanbevelingen en geoptimaliseerde codevoorstellen.  

%---------- Methodologie ------------------------------------------------------
\section{Methodologie}%
\label{sec:methodologie}

Het onderzoek volgt een toegepaste onderzoeksaanpak, waarbij een prototype-applicatie wordt ontwikkeld en getest in realistische scenario's. De methodologie omvat de volgende stappen:  

1. \textbf{Analyse van bestaande tools en bibliotheken:} Er wordt onderzocht welke linters, statische analysetools en SQL-analysetools geschikt zijn voor integratie in het prototype.  

2. \textbf{Ontwikkeling van een prototype:} De applicatie combineert tools zoals Black en Pylint voor Python-code en sqlparse voor SQL. Het prototype voert syntax- en kwaliteitscontroles uit, detecteert inefficiënte patronen en genereert overzichtelijke aanbevelingen voor verbeteringen.  

3. \textbf{Testen en evaluatie:} Het prototype wordt getest op voorbeeldprojecten met realistische code- en SQL-structuren. De effectiviteit wordt gemeten aan de hand van het aantal gedetecteerde problemen, verbeteringen in codekwaliteit en de tijdsreductie voor manuele code-review.  

4. \textbf{Deliverables en planning:} Het eindresultaat bestaat uit een werkend prototype, een evaluatierapport met analyse van resultaten en aanbevelingen voor verdere toepassing van de tool. De tijdsplanning omvat de analyse van bestaande tools (2 weken), prototypeontwikkeling (4 weken), testen en evaluatie (3 weken) en documentatie en rapportage (2 weken).  

%---------- Verwachte resultaten ----------------------------------------------
\section{Verwacht resultaat, conclusie}%
\label{sec:verwachte_resultaten}

Het onderzoek verwacht dat het prototype in staat zal zijn om veelvoorkomende problemen in zowel Python-code als SQL-query's op een reproduceerbare manier te detecteren en gestructureerde aanbevelingen voor verbetering te genereren. Voor Python-code omvat dit het opsporen van syntax- en typefouten, ongebruikte variabelen, inconsistenties in formatting en code smells zoals te complexe functies of duplicatie. Voor SQL-query's wordt verwacht dat inefficiënte joins, onnodige subquery's en ontbrekende filtervoorwaarden systematisch kunnen worden geïdentificeerd.  

De applicatie zal deze bevindingen overzichtelijk presenteren, waarbij ontwikkelaars direct inzicht krijgen in verbeterpunten en de mogelijkheid hebben om automatisch verbeterde code te bekijken. Hierdoor wordt verwacht dat de tijd die ontwikkelaars besteden aan handmatige code-review en foutopsporing aanzienlijk zal verminderen, terwijl de functionaliteit en betrouwbaarheid van de code behouden blijven.  

Het onderzoek levert een proof-of-concept op dat aantoont dat een geïntegreerde analysetool een duidelijke meerwaarde kan bieden voor softwareontwikkelaars en organisaties die streven naar efficiëntere workflows, hogere codekwaliteit en geautomatiseerde kwaliteitsbewaking. De resultaten kunnen daarnaast leiden tot verdere onderzoeksvragen, bijvoorbeeld over de toepassing van AI-gestuurde code-analyse of uitgebreidere refactoringstrategieën in complexe softwareprojecten.
